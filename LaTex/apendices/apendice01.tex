% !TeX root = ../tfg.tex
% !TeX encoding = utf8

\chapter{Apéndice}\label{ap:apendiceA}

En este apéndice se muestra la demostración del Lema~\ref{lema:raro-clasificación-gd}, extraída de~\cite{Soudry2024}.\newline

\textbf{Lema}~\ref{lema:raro-clasificación-gd}. Sea $\mathcal{L}(w)$ una función $\beta$-suave no negativa. Si $\eta < \frac{2}{\beta}$, entonces, para cualquier $w_0$ y utilizando el método del descenso de gradiente dado por:

\[
    \mathbf{w}^{(\tau + 1)} = \mathbf{w}^{(\tau)} - \eta \nabla \mathcal{L}(\mathbf{w})
\]

se tiene que $\sum_{u=0}^{\infty} \| \nabla\mathcal{L}(w^{(u)}) \|^{2} < \infty$ y, por tanto:

\[
    \lim \limits_{t \to \infty} \| \nabla\mathcal{L}(w^{(t)}) \|^{2} = 0.
\]

\begin{proof}
  Usando una propiedad conocida de las funciones $\beta$-suaves:

  \[
      | f(x) - f(y) - \nabla f(y)^{T} (x-y) | \leq \| x - y \|^2.
  \]  

  Dado que la función $\mathcal{L}(w)$ es $\beta$-suave:

  \begin{align}
      \mathcal{L} (w^{(\tau + 1)}) &\leq \mathcal{L} (w^{(\tau)}) + \nabla \mathcal{L} (w^{\tau})^{T} (w^{(\tau + 1)} - w^{\tau}) + \frac{\beta}{2} \|w^{(\tau + 1)} - w^{(\tau)}\|^2 \notag \\
      &= \mathcal{L} (w^{(\tau)}) - \eta \|\nabla \mathcal{L} (w^{(\tau)})\|^2 + \frac{\beta \eta^2}{2} \|\nabla \mathcal{L} (w^{(\tau)})\|^2 \notag \\
      &= \mathcal{L} (w^{(\tau)}) - \eta \left( 1 - \frac{\beta \eta}{2} \right) \|\nabla \mathcal{L} (w^{(\tau)})\|^2.
  \end{align}

  Así, tenemos:

  \[
      \frac{\mathcal{L} (w^{(\tau)}) - \mathcal{L} (w^{(\tau + 1)})}{\eta \left( 1 - \frac{\beta \eta}{2} \right)} \geq \|\nabla \mathcal{L} (w^{(\tau)})\|^2
  \]

  lo que implica

  \[
      \sum_{u=0}^{t} \|\nabla \mathcal{L} (w^{(u)})\|^2 \leq \sum_{u=0}^{t} \frac{\mathcal{L} (w^{(u)}) - \mathcal{L} (w^{(u+1)})}{\eta \left( 1 - \frac{\beta \eta}{2} \right)} = \frac{\mathcal{L} (w_0) - \mathcal{L} (w^{(\tau + 1)})}{\eta \left( 1 - \frac{\beta \eta}{2} \right)}.
  \]

  El lado derecho está acotado por una constante finita, dado que $\mathcal{L} (w_0) < \infty$ y $0 \leq \mathcal{L} (w^{(\tau + 1)})$. Por tanto, esto implica que

  \[
      \sum_{u=0}^{\infty} \|\nabla \mathcal{L} (w^{(u)})\|^2 < \infty,
  \]

  y, finalmente, nos queda el resultado buscado:

  \[
      \|\nabla \mathcal{L} (w^{(\tau)})\|^2 \to 0.
  \]
\end{proof}

\endinput
%------------------------------------------------------------------------------------
% FIN DEL APÉNDICE. 
%------------------------------------------------------------------------------------
