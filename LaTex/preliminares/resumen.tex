% !TeX root = ../tfg.tex
% !TeX encoding = utf8
%
%*******************************************************
% Resumen
%*******************************************************

\chapter{Resumen}

\noindent\textbf{Palabras clave:} Aprendizaje Automático, Aprendizaje Profundo, Inteligencia Artificial, Sesgo-Varianza, Aproximación No Lineal, Visión por Computador

\

\begin{comment}
El aprendizaje automático, y en particular el aprendizaje profundo, se ha convertido en una herramienta fundamental en numerosos ámbitos, desempeñando un papel determinante al complementar e incluso reemplazar las labores humanas en diversos contextos. Sin embargo, el entrenamiento de estos modelos cada vez más complejos ha revelado, recientemente, comportamientos inesperados en su rendimiento, como el fenómeno conocido como \emph{Deep Double Descent}. Este fenómeno desafía la sabiduría clásica del aprendizaje, al evidenciar que la relación entre la complejidad del modelo y su rendimiento no siempre sigue las curvas esperadas por la teoría clásica.\newline
\end{comment}


En el enfoque clásico, aumentar la capacidad de un modelo mejora su capacidad para ajustarse a los datos, pero eventualmente conduce al sobreajuste, donde el rendimiento del modelo comienza a empeorar. En contraposición, en los resultados modernos, a medida que el modelo sigue aumentando su capacidad, su rendimiento comienza a mejorar de nuevo, creando un segundo descenso que rompe con la intuición clásica y que abre una brecha entre los conocimientos teóricos y empíricos que se conocen a día de hoy. Este acontecimiento nos recuerda que el campo de la inteligencia artificial está en constante evolución y que los avances en esta disciplina no se producen de manera paralela entre la teoría y la práctica.\newline

En este TFG, nos centraremos en explicar los principios informáticos y matemáticos esenciales para comprender el fenómeno del \emph{Deep Double Descent}, tomando como base el equilibrio sesgo-varianza y la teoría de probabilidad y estadística. El objetivo principal es dar a conocer toda la información conocida hasta la fecha para explicar este fenómeno, así como la realización de diversos experimentos que sustenten la citada información con el fin de discutir las preguntas abiertas de la literatura y contribuir a la reconciliación entre la sabiduría clásica y los descubrimientos modernos.\newline

Para concluir el trabajo, se desarrollará la teoría de aproximación no lineal, destacando la analogía que presenta con el fenómeno del doble descenso y buscando ofrecer respuesta al origen del mismo en modelos neuronales suficientemente complejos.\newline