% !TeX root = ../tfg.tex
% !TeX encoding = utf8
%
%*******************************************************
% Summary
%*******************************************************

\selectlanguage{english}
\chapter{Summary}

\noindent\textbf{Keywords:} Machine Learning, Deep Learning, Artificial Intelligence, Bias-Variance, Non-Linear Approximation, Computer Vision

\

Machine learning, particularly deep learning, has become a fundamental tool in numerous fields, playing a decisive role in complementing and even replacing human tasks in various contexts. However, the training of increasingly complex models has recently revealed unexpected behaviors in their performance, such as the phenomenon known as \emph{Deep Double Descent}. This phenomenon challenges the classical wisdom of learning by showing that the relationship between model complexity and performance does not always follow the curves predicted by classical theory.\newline

In the classical approach, increasing model's capacity improves its ability to fit the data, but eventually leads to overfitting, where the model's performance starts to deteriorate. In contrast, modern findings show that as the model continues to increase its capacity, its performance begins to improve again, creating a second descent that defies classical intuition and highlights a gap between the theoretical and empirical knowledge currently available. This phenomenon reminds us that the field of artificial intelligence is constantly evolving and that advances in this discipline do not occur in parallel between theory and practice.\newline

In this Final Degree Project, we will focus on explaining the essential computational and mathematical principles needed to understand the \emph{Deep Double Descent} phenomenon, based on the bias-variance tradeoff and probability and statistics theory. The main objective is to present all the information known to date to explain this phenomenon, as well as to conduct various experiments that support this information in order to discuss the open questions in the literature and contribute to reconciling classical wisdom with modern discoveries.\newline

To conclude the project, the theory of nonlinear approximation will be developed, highlighting its analogy with the double descent phenomenon and aiming to provide an explanation for its origin in sufficiently complex neural models.\newline

\clearpage
\thispagestyle{empty}
\mbox{}
\newpage
% Al finalizar el resumen en inglés, volvemos a seleccionar el idioma español para el documento
\selectlanguage{spanish} 
\endinput
