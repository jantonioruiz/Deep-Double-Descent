% !TeX root = ../tfg.tex
% !TeX encoding = utf8

\chapter*{Glosario de notación matemática}
\addcontentsline{toc}{chapter}{Glosario de notación matemática} % Añade el glosario a la tabla de contenidos

\begin{description} 
  \item[$\mathbb{R}$] Conjunto de números reales.
  \item[$\mathbb{N}$] Conjunto de números naturales.
  \item[i.d.d.] Independientes e idénticamente distribuidas.
  \item[${\mathbb{E}[X]}$] Esperanza de la variable aleatoria $X$.
  \item[${\mathbb{E}_{\mathcal{K}}[X]}$] Esperanza de la variable aleatoria $X$ con respecto al conjunto $\mathcal{K}$.
  \item[${P[A]}$] Probabilidad de que ocurra el suceso $A$.
  \item[${P[X, Y]}$] Probabilidad conjunta de las variables aleatorias $X$ e $Y$.
  \item[$Var(X)$] Varianza de la variable aleatoria $X$.
  \item[$\mathcal{M}_{m \times n}(\mathbb{R})$] Espacio de matrices de tamaño $m \times n$ con entradas en $\mathbb{R}$.
  \item[$rang(X)$] Rango de la matriz $X$. 
  \item[$\lambda(A)$] Valor propio de la matriz $A$.
  \item[$\sigma(A)$] Valor singular de la matriz $A$.
  \item[SVD] Descomposición en valores singulares.
  \item[$X^{\dagger}$] Pseudoinversa de la matriz $X$.
  \item[OLS] Regresión lineal de mínimos cuadrados.
  \item[$H_{\mathcal{F}}$] Matriz hessiana de la función $\mathcal{F}$.
  \item[$|G|$] Cardinalidad del conjunto $G$.
  \item[$\#G$] Cardinalidad del conjunto $G$ (notación alternativa para cuando es finita).
  \item[$\nabla\mathcal{L}$] Gradiente de la función $\mathcal{L}$.
  \item[$\mathcal{DF}$] Derivada de la función $\mathcal{F}$.
  \item[$\mu-PL$] Condición modificada de Polyak y Lojasiewicz.
  \item[Esencialmente no convexo] Funciones cuya forma no es convexa en ningún entorno de un minimizador global.
  \item[$K(w)$] Núcleo tangente en el vector $w$.
  \item[$N_{eff}(A)$] Dimensionalidad efectiva de la matriz $A$.
  \item[$\langle \cdot, \cdot \rangle$] Producto interno.
  \item[$E_n(f)$] Error de aproximación lineal.
  \item[$\sigma_n(f)$] Error de aproximación no lineal.
  \item[$\mathcal{A}^{\alpha}$] Espacio de aproximación (funciones) cuyo error de aproximación $\sigma_n(f)$ está acotado superiormente por un múltiplo de $n^{-\alpha}$.
  \item[Biblioteca] En aproximación no lineal, conjunto de bases.
  \item[Diccionario] En aproximación no lineal, subconjunto arbitrario de un espacio de Hilbert.
  \item[$\sigma_n^{\mathcal{L}}(f)$] Error de aproximación con respecto a la biblioteca $\mathcal{L}$.
  \item[$\sigma_n(f, \mathbb{D})$] Error de aproximación con respecto al diccionario $\mathbb{D}$.
  \item[$\mathcal{K}_{\tau}^{o}(\mathbb{D}, M)$] Clase de funciones construidas a partir de combinaciones lineales de un conjunto finito, cuyos coeficientes están acotados por una constante que depende de $\tau$ y de $M$.
  \item[$\mathcal{K}_{\tau}(\mathbb{D})$] Unión de las clases $\mathcal{K}_{\tau}(\mathbb{D}, M)$ para $M > 0$.
\end{description}
\endinput
